\documentclass[conference]{IEEEtran}

\usepackage{cite}
\usepackage{amsmath,amssymb,amsfonts}
\usepackage{algorithmic}
\usepackage{graphicx}
\usepackage{hyperref}
\usepackage{textcomp}
\usepackage[table,xcdraw,dvipsnames]{xcolor}

\title{Data Visualization Report}
\author{\IEEEauthorblockN{Thomas Vroom}
i6291496, t.vroom@student.maastrichtuniversity.nl}

\begin{document}
\maketitle

%
% Week 1
%
\section{Chosen dataset} \noindent
I chose the dataset "\href{https://opendata.cbs.nl/#/CBS/nl/dataset/80305ned/table?ts=1762443943511}{Nabijheid voorzieningen; afstand locatie, regionale cijfers}" from the CBS as dataset for this project.
This dataset contains information on average distances to common facilities such as hospitals, schools, train stations etc.
To not make things overly complicated I only used the data from 2023 (the most recent year for which the data is complete), considered the data at municipality-scale (i.e. each municipality is its own spatial area), and used the following subset of columns: 

\noindent
\\
Avg. Distance to Closest \{GP, Pharmacy, Hospital, Supermarket, Primary School, High School, Highway, Train Station, Fire Station\}

\noindent
\\
I used the "\href{https://www.cbs.nl/nl-nl/dossier/nederland-regionaal/geografische-data/wijk-en-buurtkaart-2023}{Wijk- en buurtkaart 2023}" as the shape file for this data, also by the CBS.
This shape file uses the same municipality codes as the dataset, making the transformation from the original dataset to geospatial data very straightforward.
The dataset was preprocessed in the following way:

\begin{enumerate}
    \item Select the relevant columns from the original dataset.
    \item Remove rows that contain NaN (municipalities that do not exist anymore).
    \item Merge the shape file and the dataset on the region codes.
    \item Remove water areas from the shape file
    \item Convert the coordinate system to EPSG:4326.
    \item Export the merged shape file as GeoJSON.
\end{enumerate}

\noindent
The final dataset has 13 columns (features) and 342 rows (spatial areas; one per municipality).
































%
% Week 2
%
% \section{Multivariate visualization}

% \subsection{Multivariate sketches}
% Insert all sketches with very brief explanations of the ideas.

% \begin{figure}[h]
%     \centering
%     \includegraphics[width=0.45\linewidth]{images/examplesketch.jpg}
%     \caption{Example simple sketch and caption:"Shows the data via juxtaposed bar charts and textures."\label{fig:examplesketch}}
% \end{figure}

% \subsection{Best 3 sketches}
% Insert the best 3 sketches and discuss their advantages and disadvantages for Task 1 and the dataset. You can use \autoref{fig:exampledetailedsketch} to reference them by label.

% \begin{figure}[h]
% \centering
% \includegraphics[width=\linewidth]{images/exampledetailedsketch.pdf}
% \caption{Example detailed sketch and description for the example simple sketch. Each barchart depicts one postcode, with the barcharts positioned out in axis-aligned rows. The barchart shows the values of the selected variables using the y-position. Percentage values are kept as is, while all other variables are min-max normalized towards the minimal and maximum value present in the entire dataset. The selected variables (up to 4) are double encoded via the x-position and a texture. Possible textures are left-hatched, right-hatched, no hatch and crosshatched. The variables can be selected by clicking on one of the distributions charts present in the template. Hovering with the mouse over a bar show the detailed values of that bar. \textit{Note: This is not an effective visualization for the Task. It is purely illustrative for the amount of detail expected.}}
% \label{fig:exampledetailedsketch}
% \end{figure}

% \textbf{Advantages}
% Write about the advantage of this approach for Task 1 and dataset.

% \textbf{Disadvantages}
% Write about the disadvantage of this approach for Task 1 and dataset.

% \textbf{Chosen sketch}
% I have chosen sketch X as the sketch to be implemented. Argue why this sketch was the most suitable one for Task 1 and dataset.

% \subsection{Implementation}
% Describe what code you used from where, and how it has been adapted to match to your sketch to better suit that dataset and Task 1.

% \subsection{Performing task 1}
% Description with accompanying relevant images demonstrating how the Task can be resolved.
% %In case you did not complete it fully, show here how far you got with the implementation.

% \subsection{Self Reflection}
% Fill in the table with your expected point total. Write down what you believe is missing.

% \begin{table}[b]{
%     \begin{tabular}{|m{0.08\linewidth}|m{0.82\linewidth}|}
%     \hline
%     \rowcolor[HTML]{EFEFEF} 
%     Points & Sub component\\ \hline
%     \rowcolor[HTML]{DAE8FC} \hfill ? & There are at least 10 different rough sketches of nD visualizations that could potentially solve Task 1. \\ \hline
%     \rowcolor[HTML]{EFEFEF} \hfill ? & There are at least 3 different rough sketches of nD visualizations not shown directly in the lecture that could potentially solve Task 1.\\ \hline
%     \rowcolor[HTML]{DAE8FC} \hfill ? & The three different detailed sketches can solve Task 1.\\ \hline
%     \rowcolor[HTML]{EFEFEF} \hfill ? & The detailed sketches could be implemented without confusion by a third-party. \\ \hline
%     \rowcolor[HTML]{DAE8FC} \hfill ? & The tradeoffs of the detailed sketches for Task 1 are discussed well.\\ \hline
%     \rowcolor[HTML]{EFEFEF} \hfill ? & The chosen visualization sketch can solve Task 1 well. \\ \hline
%     \rowcolor[HTML]{DAE8FC} \hfill ? & It is well argued why the chosen visualization sketch was chosen.\\ \hline
%     \rowcolor[HTML]{EFEFEF} \hfill ? & The final visualization sketch is fully implemented and suitable for Task 1. \\ \hline
%     \rowcolor[HTML]{DAE8FC} \hfill ? & Labels, colors and other design elements have been included in the implementation and are suitable.\\ \hline
%     \rowcolor[HTML]{EFEFEF} \hfill ? & Interactivity for filtering and selecting variables has been built into the visualization and is fully functional\\ \hline
%     \rowcolor[HTML]{DAE8FC} \hfill ? & Task 1 has been performed well with the implemented visualization.\\ \hline
%     \rowcolor[HTML]{EFEFEF} \hfill ? & You have self-reflected on this rubric, and filled in your expected amount of points for each sub component.\\ \hline
%     \end{tabular}%
%     }
% \end{table}

% %
% % Week 3
% %
% \section{Spatial visualization}

% \subsection{Spatial sketches}
% Insert all sketches with very brief explanations of the ideas.

% \subsection{Best 3 sketches}
% Insert the best 3 sketches and discuss their advantages and disadvantages for the task and dataset. Can use \autoref{fig:exampledetailedsketch} to reference them by label.

% \textbf{Advantages}
% Write about the advantage of this approach for Task 2 and dataset.

% \textbf{Disadvantages}
% Write about the disadvantage of this approach for Task 2 and dataset.

% \textbf{Chosen sketch}
% Argue why this sketch was the most suitable one for Task 2 and dataset.

% \textbf{Implementation}
% Describe what code you used from where, and how it has been adapted to match to your sketch to better suit that dataset and task.

% \subsection{Performing Task 2}
% Description with accompanying relevant images demonstrating how Task 2 can be resolved.
% %In case you did not complete it fully, show here how far you got with the implementation.

% %
% % Week 4
% %
% \section{Linked Clustering}

% \subsection{Rough Interaction to Multivariate sketches}
% Insert your rough sketches and argue which one is most suitable for Task 3 and the dataset.

% \subsection{Rough Interaction to Spatial sketches}
% Insert your rough sketches and argue which one is most suitable for Task 3 and the dataset.

% \subsection{K-means Interactions}
% Insert your rough sketches and argue which one is most suitable for Task 3 and the dataset.

% \subsection{Self Reflection}
% Fill in the table with your expected point total. Write down what you believe is missing.

% \begin{table}[b]{
%     \begin{tabular}{|m{0.08\linewidth}|m{0.82\linewidth}|}
%     \hline
%     \rowcolor[HTML]{EFEFEF} 
%     Points & Sub component \\ \hline
%     \rowcolor[HTML]{DAE8FC} \hfill ? & There are at least 10 different rough sketches of spatial visualizations that could potentially solve Task 2. \\ \hline
%     \rowcolor[HTML]{EFEFEF} \hfill ? & There are at least 3 different sketches of spatial visualizations not shown directly in the lecture that could potentially solve Task 2. \\ \hline
%     \rowcolor[HTML]{DAE8FC} \hfill ? & The three different detailed sketches can solve Task 2.\\ \hline
%     \rowcolor[HTML]{EFEFEF} \hfill ? & The detailed sketches could be implemented without confusion by a third-party. \\ \hline
%     \rowcolor[HTML]{DAE8FC} \hfill ? & The tradeoffs of the detailed sketches for Task 2 are discussed well.\\ \hline
%     \rowcolor[HTML]{EFEFEF} \hfill ? & The chosen visualization sketch can solve Task 2 well. \\ \hline
%     \rowcolor[HTML]{DAE8FC} \hfill ? & It is well argued why the chosen visualization sketch was chosen.\\ \hline
%     \rowcolor[HTML]{EFEFEF} \hfill ? & The final visualization sketch is fully implemented and suitable for Task 2. \\ \hline
%     \rowcolor[HTML]{DAE8FC} \hfill ? & Labels, colors and other design elements have been included in the implementation and are suitable.\\ \hline
%     \rowcolor[HTML]{EFEFEF} \hfill ? & Interactivity for filtering and selecting variables, \textbf{and selecting spatial areas} has been built into the visualization, is fully functional and has low cost of interaction. \\ \hline
%     \rowcolor[HTML]{DAE8FC} \hfill ? & Task 2 has been performed well with the implemented visualization.\\ \hline
%     \rowcolor[HTML]{EFEFEF} \hfill ? & You have self-reflected on this rubric, and filled in your expected amount of points for each sub component.\\ \hline
%     \end{tabular}%
%     }
% \end{table}

% \subsection{Implementation}
% Describe what code you used from where, and how it has been adapted to match to your sketch to better suit that dataset and Task 3.

% \subsection{Resolving Task 3}
% Description with accompanying relevant images demonstrating how Task 3 can be resolved.
% %In case you did not complete it fully, show here how far you got with the implementation.

% \subsection{Self Reflection}
% Fill in the table with your expected point total. Write down what you believe is missing.

% \begin{table}[h!]{
%     \begin{tabular}{|m{0.08\linewidth}|m{0.82\linewidth}|}
%     \hline
%     \rowcolor[HTML]{EFEFEF} 
%     Points & Sub component \\ \hline
%     \rowcolor[HTML]{DAE8FC} \hfill 2 & There are at least 3 different rough sketches of how to link the output of the clustering to the nD visualization. \\ \hline
%     \rowcolor[HTML]{EFEFEF} \hfill 1 & The chosen interaction of the output of the clustering to the nD visualization is well argued for. \\ \hline
%     \rowcolor[HTML]{DAE8FC} \hfill 2 & There are at least 3 different rough sketches of how to link the output of the clustering to the spatial visualization. \\\hline
%     \rowcolor[HTML]{EFEFEF} \hfill 1 & The chosen interaction of the output of the clustering to the spatial visualization is well argued for. \\ \hline
%     \rowcolor[HTML]{DAE8FC} \hfill 2 & There are at least 3 different rough sketches of how to interact with k-means. \\ \hline
%     \rowcolor[HTML]{EFEFEF} \hfill 1 & The chosen interaction with k-means is well argued for. \\ \hline
%     \rowcolor[HTML]{DAE8FC} \hfill 5 & The final visualization sketches are fully implemented and are suitable for Task 3. \\ \hline
%     \rowcolor[HTML]{DAE8FC} \hfill 4 & The clustering algorithm can be interacted with in a suitable manner. \\ \hline
%     \rowcolor[HTML]{EFEFEF} \hfill 4 & The clustering is linked well to the multidimensional visualization through interactions. \\ \hline
%     \rowcolor[HTML]{DAE8FC} \hfill 4 & The clustering is linked well to the spatial visualization through interactions. \\ \hline
%     \rowcolor[HTML]{DAE8FC} \hfill 3 & Task 3 has been performed with the implemented visualization. \\ \hline
%     \rowcolor[HTML]{EFEFEF} \hfill 1 & You have self-reflected on this rubric, and filled in your expected amount of points for each sub component. \\ \hline
%     \end{tabular}%
%     }
% \end{table}

\end{document}
